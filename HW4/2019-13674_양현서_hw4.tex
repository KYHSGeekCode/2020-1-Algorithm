\documentclass{article}
\usepackage{graphicx}
\usepackage{kotex}
\usepackage{lipsum}
\usepackage{listings}
\usepackage{graphicx}
\usepackage[a4paper, total={6in, 8in}]{geometry}
\usepackage{float}
\usepackage{amsmath}

\title{알고리즘 H/W \#4}
\date{}
\author{
  2019-13674\\
  양현서 \\
  바이오시스템소재학부\\
}
\lstset{
    language=C, %% Troque para PHP, C, Java, etc... bash é o padrão
    basicstyle=\ttfamily\small,
    numberstyle=\footnotesize,
    numbers=left,
    frame=single,
    tabsize=2,
    title=\lstname,
    escapeinside={\%*}{*)},
    breaklines=true,
    breakatwhitespace=true,
    framextopmargin=2pt,
    framexbottommargin=2pt,
    extendedchars=false,
    inputencoding=utf8
}

\begin{document}
\maketitle


\section{숲집합은 매트로이드를 이룬다고 했다. 숲은 사이클을 이루지 않는 간선들의 집합으로 정의할 수 있다. 여기 제약을 조금 풀어서 사이클을 최대 하나까지만 허용하는 아래 간선들의 집합 ‘근사숲’은 매트로이드가 되는가?}
근사숲 = $\{E' \subseteq E | E' \text{의 간선들이 사이클을 만들지 않거나 최대 1개까지의 사이클을 갖는다}\}$

\subsection{상속성}
근사숲의 원소의 부분집합은 최대 1개까지의 사이클을 가지거나 사이클을 가지지 않고 숲도 상속성을 가지므로 근사숲도 상속성을 가진다.
\subsection{증강성, 교환성}
$|A| < |B|$인 두 근사숲 A, B를 생각하자. A는 적어도 두 개 이상의 분리된 근사트리를 가진다. A에 속하는 임의의 근사 트리를 T라 하자.
\\
먼저 T에 사이클이 있는 경우, T의 정점들에 대해 이들을 연결하는 간선 수는 B가 A보다 많을 수 없다. 그렇지 않으면 B가 사이클을 2개 이상 가지게 된다. A의 다른 트리에 대해서도 마찬가지이다. B가 A보다 간선수가 많으므로 B에는 A의 서로 다른 근사트리를 연결하는 간선이 적어도 하나 이상 존재하게 된다. 이 간선들 중 하나를 A에 더하면 사이클을 새로 만들지 않아 역시 근사숲이 된다.
\\
A에 사이클이 하나도 없는 경우, T의 정점들에 대해 이들을 연결하는 간선의 수는 B가 A 보다 1개 초과 많을 수는 없다. B가 A보다 1개 많을 경우 B에는 사이클이 1개 생성되며, 이 간선을 A에 추가하면 A에 사이클이 1개 생성되므로 역시 근사숲이 된다. 만약 B가 A보다 많은 간선을 가지지 않는다면, B는 A보다 간선이 많으므로 B에는 A의 서로 다른 근사트리를 연결하는 간선이 적어도 하나 이상 존재하게 된다. 이 간선들 중 하나를 A에 더하면 사이클을 1개 넘게 만들지 않아 역시 근사숲이 된다. 따라서 근사숲은 증강성, 교환성을 가진다.

따라서 근사숲도 매트로이드이다.

\section{BoyerMooreHorspool 알고리즘으로 스트링 매칭을 하려한다. 총 10,000개의 문자로 이루어진 텍스트에서 “camouflage"라는 패턴을 검색하려 할 때 다음의 질문에 답하라.(asymptotic number가 아니고, 정확한 숫자를 요구함.)}
\subsection{가장 운이 좋으면 텍스트에서 몇 개의 문자를 살펴보고 끝날 수 있는가?}
텍스트가 camouflage에 있는 글자가 아닌 글자로만 이루어질 경우 계속 매칭이 한글자 비교만에 실패하고 점프하게 되므로 10000/10 = 1000개의 문자만 살펴보면 되게 된다.
\subsection{가장 운이 나쁘면 텍스트에서 몇 개의 문자를 살펴보고 끝날 수 있는가?}
계속 매칭에 성공할 경우 모든 글자를 전부 살펴보아야 한다. 10000개

\section{2SAT은 다음과 같이 정의된다. $2SAT \in P$ 임을 증명하라.}
$$2SAT = \{ \Phi | \Phi \text{ is a satisfiable formula in CNF with at most 2 literals per clause}\}$$
Hint: A clause $x \lor y$ can be written as $\neg x \Rightarrow y$ or as $\neg y \Rightarrow x$. Consider a graph whose vertices are variables and their complements, and where there is a directed edge from $\neg x to y$ if x and y are in the same clause.

2CNF를 P 시간에 풀 수 있는 알고리즘을 찾으면 2SAT는 P이다.
\begin{enumerate}
    \item 먼저, 각 CNF들을 변수는 정점, 논리연산자는 $(a \lor b) \Rightarrow \neg a \to b, \neg b \to a$ 의 유향간선으로 변환한다. 이 그래프는 각 변수 n개에 대해 원본과 그 부정, 총 2n개의 정점을 갖게 된다.
    \item 이 그래프에서 어떤 정점 a에서 출발하여 정점 $\neg a$로 도착하는 경로가 존재한다면, 이 식은 모순이 있으므로 답이 거짓임을 알 수 있다. a와 $\neg a$가 동시에 참이거나 동시에 거짓일 수는 없기 때문이다.
    \item 반대로, 어떤 정점 $\neg a$에서 출발하여 정점 a로 도착하는 경로가 존재한다면, 이 식은 모순이 있으므로 답이 거짓임을 알 수 있다. a와 $\neg a$가 동시에 참이거나 동시에 거짓일 수는 없기 때문이다. 
    \item 어떤 정점 a를 택하여 참을 대입한다. a에서 출발하여 도달할 수 있는 모든 정점에 참을 대입한다. 그리고 그 대입되는 정점의 부정에는 거짓을 대입한다.
    \item 바로 앞 단계를 모든 정점이 값이 채워질 때까지 반복한다.
    \item 위 두 단계의 수행 횟수는 기껏해야 $O(n^2)$이다.
    \item 중간에 모순 없이 모든 정점을 채웠다면 이 2SAT를 만족시키는 변수들의 쌍을 구한 것이다.
    \item 따라서 2SAT는 P이다.
\end{enumerate}


\section{HAM-PATH 문제는 아래와 같이 정의된다.}

Input: a graph $G=(V, E)$, two vertices u and v
Question: Is there a Hamiltonian path from u and v ?

Definition: Hamiltonian path- a simple path that visits every vertex exactly once

HAM-CYCLE이 NP-Complete임을 이용하여 HAM-PATH 문제가 NP-Hard임을 증명하여라.
\\
 주어진 그래프 G에 정점 w를 추가한다. 이 정점 w를 그래프 G의 모든 정점과 연결한다. 이 결과를 그래프 H라 하자. 이제 그래프 H에 대해 정점 w에서 출발하는 Hamiltonian cycle 알고리즘을 적용한다. w에서 출발하여 H의 모든 정점을 한 번씩만 거쳐 w로 돌아오는 path를 얻는다. 이 때 이 임의의 cycle에서 w 다음으로 처음 만나는 정점 u, w로 돌아오기 직전 정점을 v라 하면 이 path에서 정점 w를 제거한 u, v 사이의 hamiltonian path가 존재한다. 이 정점 w는 그래프 G의 모든 정점과 연결되어 있으므로 대칭성에 의해 임의의 u, v에 대해서도 성립한다. 만약 hamiltonian cycle이 존재하지 않으면, u, v 사이의 hamiltonian path가 존재하지 않는다. 만약 hamiltonian cycle이 존재하지 않음에도 w를 제외했을 때 u, v 사이의 hamiltonian path가 존재한다면 그 u, v 사이에 정점 w를 추가한 그래프 H에서 w에서 출발하는 hamiltonian cycle이 존재하므로 모순이다.
 따라서 hamiltonian path 문제는 hamiltonian cycle 문제로 풀 수 있고, hamiltonian cycle 문제는 NP-Complete, 즉 NP-Hard이므로 hamiltonian path 문제도 NP-Hard이다.

\section{CLIQUE이 NP-Complete임을 이용하여 아래의 SUBGRAPH-ISOMORPHISM 문제가 NP-Complete임을 증명하라.}
SUBGRAPH-ISOMORPHISM

Input: Two graphs, $G=(V_1, E_1)$ and $H=(V_2, E_2)$

Question: Does G contain a subgraph isomorphic to H, that is, a subset $V \subseteq V_1$ and a subset $E \subseteq E_1$ s.t. $|V|=|V_2|, |E|=|E_2|$, and there exists a 1-to-1 function $ f : V_2 \to V$ satisfying $\{u,v\} \in E_2$ iff $\{f(u),f(v)\} \in E $?

Hint: 이것은 아주 쉬운 문제로 여러분이 NP-Complete의 증명 process를 가장 단순하게 연습할 수 있도록 하는 것이 목적임.

\subsection{NP}
H와 동형인 G의 서브 그래프라고 주장되는 I가 주어졌을 때 이 그래프 I가 G의 서브 그래프이고 H와 동형인지 확인하는 것은 다항시간 안에 확인할 수 있다. 따라서 SUBGRAPH-ISOMORPHISM은 NP이다.

\subsection{NP-Hard}
H의 크기가 k라고 하자. 그래프 G에서 크기 k인 complete subgraph가 존재한다면, 즉 이 그래프에 대해 크기 k인 CLIQUE를 적용하여 그 complete subgraph가 존재한다면, 이 complete subgraph에는 H와 isomorphic한 subgraph가 반드시 존재한다. 만약 그래프 G에서 크기가 k인 complete subgraph가 존재하지 않는다면 그래프 G의 subgraph 중 H와 isomorphic subgraph가 존재하지 않는다. 따라서 CLIQUE 문제를 풀면 SUBGRAPH-ISOMORPHISM도 풀 수 있고, SUBGRAPH-ISOMORPHISM 문제는 CLIQUE 문제로 다항식 시간(기껏해야 $O(K^2)$) 안에 변환할 수 있다. CLIQUE 문제가 NP-Complete (즉, NP-Hard)이므로 SUBGRAPH-ISOMORPHISM도 NP-Hard이다.

\subsection{NP-Complete}
따라서 SUBGRAPH-ISOMORPHISM은 NP-Complete이다.

\section{Triangle inequality를 만족하지 않는 TSP의 경우에는 P=NP가 아닌 한 어떠한 상수 ρ에 대해서도 최적해의 ρ배를 넘지 않는 해를 보장할 수 있는 polynomial-time 알고리즘이 존재하지 않는다는 사실을 배웠다. 이제 다음의 경우를 생각해 보자.}
\subsection{도시들간의 거리가 ‘1부터 100까지의 정수’ 값 중에 ‘임의로(randomly)’ 정해진 TSP가 있다. 즉, Triangle inequality를 만족하지 않는다. Triangle inequality를 만족하지 않지만 도시간의 거리가 일정한 상수 범위 안에 있다. 이러한 성질을 만족하는 TSP에 대해서, 여러분이 배운 TSP를 위한 근사 알고리즘을 이용하여 어느 정도까지 품질을 보장할 수 있는 방법이 있다. 이 방법을 밝히고 보장할 수 있는 품질은 어느 정도인지 이야기하라.}
간선들의 가중치가 1부터 100까지의 정수이므로, 모든 간선의 가중치에 98을 더하면 삼각부등식을 만족하는 그래프가 된다. 이 때 수업 시간에 배운 알고리즘으로 최적해의 3/2배를 초과하지 않는 해를 구할 수 있으며, 이 해에 98*(간선의 개수)를 원래 문제의 해를 구할 수 있다. 따라서 간선의 개수를 n이라 하면, 이 문제에서 최적해가 M 이라 하면 $\frac 32 (M + 98 n)$ 를 초과하지 않는 해를 구할 수 있다.


\subsection{위의 아이디어를 최단경로 문제에 적용하는 것에 대해 생각해보자. Bellman-Ford 알고리즘은 weight이 음인 edge를 허용하지만 Dijkstra 알고리즘보다 시간이 많이 걸린다. 그러면 위처럼 가중치가 가장 작은 음수보다 큰 어떤 수를 모든 edge의 가중치에 더해서 가중치가 모두 양인 문제로 만들면 Dijkstra 알고리즘을 적용할 수 있지 않을까 하는 생각을 할 수 있다. 이에 대한 여러분의 생각을 적어보라.}
그럴 듯한 생각이다. 그러나 정점 A에서 정점 B로 가는 path가 I, J 두 개가 존재하고, I의 경로의 길이가 J의 경로의 길이보다 짧다고 할 때, 원본 그래프에서는 I가 J보다 짧은 경로라고 생각해야 하지만, 만약 모든 edge에 양수를 더해주면, 이후 경로의 길이는 각 경로에 포함된 edge의 개수에 의존하여 변하게 된다. 따라서 만약 path I가 더 많은 간선을 포함하고 있었다고 하면, path I 에 더해지는 양수의 횟수가 path J에 더해지는 양수의 횟수보다 많아지게 되어, '양수 더하기 변환' 이후의 path I'가 path J'보다 길다고 판단하게 되는 경우가 생길 수 있다. 이는 의도와 달리 원래 path I가 path J보다 짧다는 판단과 다르게 되며, 원래 원하던 답과 달라질 수 있다. 따라서 이 방법은 틀린 결론을 낼 위험이 있다.
\end{document}

% \begin{align*}
%     T(n) &\le T(n/4) + T(3n/4) + \Theta(n) \\
%          &\le \frac{cn}{4}\log(n/4) + \frac{3cn}{4}\log(3n/4) + \Theta(n) \\
%          &=\frac{cn}{4}(\log n- \log 4 + 3 \log 3 + 3 \log n - 3\log 4)+ \Theta(n) \\
%          &=cn\log n - \frac{cn}{4} \log \frac{3^3}{4^4} + \Theta(n) \\
%          &\le cn \log n 
% \end{align*}
